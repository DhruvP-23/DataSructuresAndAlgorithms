\documentclass{article}

\title{Run Times of Tree-Based Data Structures}
\author{Your Name}

\begin{document}

\maketitle

\section{Introduction}

In this document, we will discuss the run times of various tree-based data structures commonly used in computer science.

\section{Binary Search Tree (BST)}

A Binary Search Tree is a binary tree where the value of each node is greater than all values in its left subtree and less than all values in its right subtree. The run times for common operations on a BST are as follows:

\begin{itemize}
    \item Insertion: $O(\log n)$
    \item Deletion: $O(\log n)$
    \item Search: $O(\log n)$
\end{itemize}

\section{AVL Tree}

An AVL Tree is a self-balancing binary search tree where the heights of the left and right subtrees of any node differ by at most one. The run times for common operations on an AVL tree are as follows:

\begin{itemize}
    \item Insertion: $O(\log n)$
    \item Deletion: $O(\log n)$
    \item Search: $O(\log n)$
\end{itemize}

\section{Red-Black Tree}

A Red-Black Tree is another self-balancing binary search tree where each node has an extra bit representing its color. The run times for common operations on a Red-Black tree are as follows:

\begin{itemize}
    \item Insertion: $O(\log n)$
    \item Deletion: $O(\log n)$
    \item Search: $O(\log n)$
\end{itemize}

\section{B-Tree}

A B-Tree is a self-balancing search tree that can have more than two children per node. The run times for common operations on a B-Tree are as follows:

\begin{itemize}
    \item Insertion: $O(\log n)$
    \item Deletion: $O(\log n)$
    \item Search: $O(\log n)$
\end{itemize}

\section{Conclusion}

In this document, we have discussed the run times of various tree-based data structures commonly used in computer science. It is important to consider these run times when choosing the appropriate data structure for a given problem.

\end{document}